\begin{tabular}{>{\raggedright\arraybackslash}p{3cm}>{\raggedright\arraybackslash}p{4cm}p{5cm}}
\toprule
Property  & Technique & Description \\ 
\midrule
Surface chemistry & Energy dispersive  X-ray spectroscopy (EDS) & Samples are exposed to beam of X-rays, allowing elements to be identified as they excite and emit unique fluorescence X-rays.  \\ 
 & Fourier transform infrared spectroscopy (FTIR) & The amount of infrared light absorbed in a sample by molecular bonds is used to infer what bonds are present and how biosorption alters these bonds. \\ 
 & Raman spectroscopy (RS) & Visible or infrared light is is passed through a sample. The amount of light scattered can be used to identify molecular bonds. \\ 
 & Potentiometric titration & Point of zero charge (pH\textsubscript\{pzc\})  determined where there is an absence of adsorbent surface charge. \\ 
 & Electrophoresis & An electric field is applied in combination with light scattering to determine the zeta potential, defined as the potential difference at the interface of the diffusion layer of the adsorbent and bulk fluid used as a measure of electrostatic repulsion. \\ 
Morphology & Scanning electron microscopy (SEM) & An electron beam scans across the adsorbent surface, resulting in scattered electrons that are used to generate an image. \\ 
 & Transmission electron microscope (TEM) & A high powered beams of electrons is sent through a cross section of a sample, resulting in interactions with the sample that are used to generate an image.  \\ 
 & Laser diffraction & Samples are exposed to a laser beam. The patterns of diffraction are analysed and used to infer particle size. \\ 
Hydrophobicity & Water contact angle & The shape of the liquid-vapour interface of water on the sample is used to determine surface wettability. \\ 
Textural property & BET specific surface area & The volume of \ce\{N\_2\} adsorbed on the adsorbent surface can be used to infer surface are. \\ 
Thermal stability & Thermal gravimetric analysis & The mass of a sample is recorded as a function of temperature and time to quantitatively analyse composition. \\ 
Crystalline structure & X-ray diffraction & The diffraction pattern that arises from a sample being exposed to an X-ray beam is used to identify crystalline compounds. \\ 
Proximate analysis & ASTM international standards & A variety of standardised methods employed to determine the distribution of major constituents. \\ 
Ultimate analysis  & ASTM international standards & A variety of standardised methods employed to determine the distribution of carbon, hydrogen, oxygen and nitrogen. \\ 
Adsorption property & Iodine number & The mass of iodine consumed by the adsorbent is used to compare adsorption surface area and porosity. \\ 
 & Molasses number & The measure of the amount of colour removed from molasses by an adsorbent and is used to evaluate the macro pore structure. \\ 
 & Methylene blue index & The measure of the amount of methylene blue adsorbed by a sample for the purpose of determining the cationic exchange capacity. \\
 \bottomrule 
\end{tabular}
